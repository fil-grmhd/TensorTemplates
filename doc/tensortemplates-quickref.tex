\documentclass{article}
\usepackage{amsmath,hyperref,enumitem,braket}

\begin{document}

\title{TensorTemplates: Quick reference} 
\author{SvenK}

\maketitle

\begin{abstract}
This is a first approach for a quick
reference for the TensorTemplates code,
available at
\url{https://bitbucket.org/infrastructuredevs/tensortemplates}
\end{abstract}

\section{Basics}

Mathematically, a tensor is defined as $T^{a,b,c,\dots}_{i,j,k,\dots}$ with two independent sets of covariant and contravariant indices, i.e. the index lists have an order only within the list of covariant and the list of contravariant indices However, sometimes it is perferable to define a total order of indices. This is what TensorTemplate does. Such an object could be written as $T_a^{~b} {~}_c^{~d}$ instead of $T_{ac}^{bd}$. Especially, this attemp allows to write \texttt{T(a,b,c,d)}, resembling the notation and idea of a simple n-dimensional array.

We therefore define a Tensor in TensorTemplates as an n-dimensional list where each index has a \emph{type} which defines it either as an upper or lower index (covariant or contravariant, respectively).

\section{Short reference}
In this reference, each index $x_i$ is
associated with a type. The type decides whether
the index shall be up or down. The fact that
the indices are put as lower indices in the
notation of this section is irrelevant.

\begin{description}[style=nextline]
\item[\texttt{contract$\braket{i,j}(T,U)$}]
Does the contraction of the $i$'s index of $T$ with the $j$'s index of $U$. That is,
\begin{align*}
&\phantom{=}
\mathtt{contract}\braket{i,j}\left(
   T_{a_1,a_2,a_3,\dots,a_N},
   U_{b_1,b_2,b_3,\dots,b_M}
\right) \\
&=
\sum_c
T_{a_1,\dots,a_{i-1},c,a_{i+1},\dots}
U_{b_1,\dots,b_{j-1},c,b_{i+1},\dots}
\\
&=
V_{a_1,\dots,a_{i-1},a_{i+1},\dots,a_{N-1},
   b_1,\dots,b_{j-1},b_{j+1},\dots,b_{M-1}}
\end{align*}
Obviously, this is defined similiar to the
mathematical contraction.

\item[\texttt{trace$\braket{i,j}(T)$}]
Computes the trace within a tensor. Note that
since
\texttt{contract$\braket{i,j}(T,T)$)}$=T_i T_j$,
there is a need to express $T^i_j$ which is
of course a different operation. The trace
between to indices is defined as
\begin{align*}
&\phantom{=}
\mathtt{trace}\braket{i,j}\left(
   T_{a_1,a_2,a_3,\dots,a_N},
\right) \\
&=
\sum_c
T_{a_1,\dots,a_{i-1},c,a_{i+1},\dots,a_N}
\\
&=
V_{a_1,\dots,a_{i-1},a_{i+1},\dots,a_{N-1}}
\end{align*}

\item[\texttt{tensor\_cat($T$,$U$)}]
Computes the outer product between the tensors.
That is,
\begin{equation*}
\mathtt{tensor\_cat}(
  T_{a_1,\dots,a_N},
  U_{b_1,\dots,b_N}
) = V_{a_1,\dots,a_N,b_1,\dots,b_N}
\end{equation*}
Similiar as one would expect from the outer
product.

\item[\texttt{reorder\_index<$i$,$j$>($T$)}]
This operation swaps the axes $i$ and $j$.
It is defined as
\begin{equation*}
\mathtt{reorder\_index}\braket{i,j}\left(
T_{a_1,\dots,a_i,\dots,a_j,\dots,a_N}
\right)
= T_{a_1,\dots,a_j,\dots,a_i,\dots,a_N}
\end{equation*}

\end{description}

\section{What I am missing}

The possibility to contract two or multiple
indices in one operation without a chain
of \texttt{trace} and \texttt{contract}
operations, i.e. I want to compute $D$ in

\begin{equation*}
A^{ik} B^{nm} C_{knm}
= D^i
\end{equation*}
%
Then I can do this with
%
\begin{equation*}
D=
\mathtt{contract}\braket{1,0}\left(A,
\mathtt{trace}\braket{0,2}\left(
\mathtt{contract}\braket{0,2}\left(
B,C
\right)\right)\right)
\end{equation*}
%
However this is really not well readable.
Instead, one should be able to write something
like this pseudocode:
\begin{equation*}
D = 
\mathtt{contract}\braket{1,0}\left(A,
\mathtt{contract}\braket{0,0}\braket{1,1}(B,C)
\right)
\end{equation*}
A nice way of writing this down would probably
be this pseudocode
\begin{equation*}
D=
\mathtt{trace}\braket{2,5}\braket{2,4}\braket{3,6}
\mathtt{tensor\_cat}\left(A,B,C\right)
\end{equation*}
because it avoid the headache of computing the
indices of the intermediate results in the head.

What is the TensorTemplate'y way of doing this?



\end{document}